\chapter{Installing Libmoleculizer}
\label{chap:installingChapter}

\section{Compiling and installing from source}
The libmoleculizer package uses the Gnu Autotools to setup the build
(this is the familiar configure, make, and make install style
compilation) and should be fairly easy to install on a wide variety of
platforms.

\subsection{Prerequisites}
Libmoleculizer has a single prerequisites which are necessary to compile and
run libmoleculizer, a version of the libxml++ library.   Any version
should be able to be detected and used properly with libmoleculizer.  

The easiest way to install libxml++ is to use your systems package
manager, if you have one, which will ensure that all the necessary
files are installed and configured properly, in an automatic fashion.
For instance, on a Ubuntu linux system, entering the command ``sudo
apt-get install libxml++2.6-dev'' or ``sudo apt-get install
libxml++1.0-dev'', depending on which version you prefer (version 1.0
is simpler and smaller, version 2.6 is larger and has more
dependencies, but has unicode support).  On a Fedora linux system, the
command ``sudo yum install libxml++'' should work.

On Macintosh systems, there is no standard package manager that comes
with the system.  MacPorts (http://www.macports.org/) and Fink
(http://www.finkproject.org/) are installable package managers for Mac
systems, both of which provide libxml++.  Using Fink, either of the
commands ``sudo fink install libxml++1'' or ``sudo fink install
libxml++2'' will work, depending on which version you wish to install
(1 is much smaller; 2 has unicode support).  Using MacPorts, the same
corresponding commands would be either ``sudo port install libxmlxx''
or ``sudo port install libxmlxx2'', which will install either version
1 or 2 respectively.


\subsubsection{Manually compiling and installing  libxml++}
If you do not have a package manager, or just wish to go it alone, the
next option is to compile libxml++ from source.  Source packages for
libxml++ can be obtained from the downloads page of the libxml++
webisite, which can be found at
http://libxmlplusplus.sourceforge.net//  If this is the case, 
using version 1.0, as opposed to any of the 2.x versions, is HIGHLY
recommended, as libxml++-1.0 has a single prerequisite, whereas
libxml++-2.x has legion. Installing 2.x from source is possible, and
not terribly tricky, but is very time consuming, and will not be
discussed here.

To install libxml++-1.0 from scratch, another library, libxml2, must
first be compiled and installed.  As it may already be installed on
your system, attempt to run ./configure from within the libxml++-1.0
directory.  If it works, great: followup with running make and then
make install and you're done.  If not, follow the steps in the next
paragraph to ensure libxml2 is installed.

Libxml2 (http://xmlsoft.org/) can be downloaded from
ftp://xmlsoft.org/libxml2/, At the time of this writing, the latest
version is 2.7.3.  It has no prerequisites and so simply running
./configure, make, and make install should be enough to install
libxml2.

Once this is done, libxml++-1.0 can be compiled, using a similar
./configure, make, and make install procedeure.  

After this is done, all the necessary preconditions for installing
libmoleculizer have been satisfied.

\subsection{Simplest compile/installation Procedure}
Depending on your system configuration, the following procedure will
probably be all that is necessary to get a working version of
libmoleculizer onto your system.

First, obtain the source.  The best option for this is to go to
\libmzrwebsite and download the libmoleculizer-\currentversion.tar.gz source
file, and make sure it is unpacked.  

Assuming everything is setup, and the prerequisites are met, then
installation will likely be as easy as going into the source directory
and entering the following commands, which will compile the source
code into an executable and install it in a place where it is
globally accessable.  

\begin{Shell}
./bootstrap.sh
./configure
make
sudo make install
\end{Shell}

\subsection{Can't or Dont want to install libmoleculizer globally}
By default {\bf ./configure} installs libmoleculizer to a global
location, usually /usr/lib or /usr/local/lib depending on your
system.  This may either be unacceptable or impossible, say if you do
not have administrator access on your computer.  This can be fixed py
passing a {\bf --prefix=\emph{location}} flag to the {\bf ./configure}
command.  

\subsection{./configure cannot find necessary libraries}
Configure may fail and say that libxml++ cannot be found.  The first
thing to do, of course, is make sure it is installed on
your machine.  If it isn't, that's the problem.  Follow the directions
in the Prerequisites section and repeat.

If it is installed, the likely problem is that pkg-config cannot find
libxml++.  



If they are, then configure must be explicitly told
where to find the libraries and/or the corresponding header files. The
easiest way to do this is to pass the header file locations to the
CXXFLAGS environmental variable, and the library locations to the
LDFLAGS environmental variable.

\subsection{Enabling the demos}
Libmoleculizer has several demos that come with it, that demonstrate
in principle how to design a particle simulator using libmoleculizer
(particlesim\_demo), how to design a stochastic simulator a la
Gillespie using libmoleculizer (stochasticsim\_demo), how to use
libmoleculizer to expand reaction networks (network\_expander\_demo),
and how to use the c interface (c\_interface\_demo).  These examples are
located in the doc/demos/simulators/ directory, however, by default,
they are neither built nor installed.  

If you desire to use these programs (recommended when learning to use
libmoleculizer), add the flag --enable-demos when calling configure,
i.e. run ``./configure --enable-demos''.  

